\documentclass[a4paper, 12pt, oneside]{book}

\usepackage[T1]{fontenc}
\usepackage[utf8]{inputenc}
\usepackage[brazilian]{babel}
\usepackage[top = 2cm, bottom = 2cm, left = 2.5cm, right = 2.5cm]{geometry}
\usepackage{amsfonts}
\usepackage[dvipsnames, svagnames]{xcolor}
\usepackage{multicol}
\usepackage{amsmath, array, amssymb}
\usepackage{graphicx}
\linespread{1.5}

\begin{document}

\begin{titlepage}

\addtolength{\topmargin}{1.5cm}

\setlength{\baselineskip}{2.0\baselineskip}

\begin{center}

{\large{UNIVERSIDADE FEDERAL DE SÃO CARLOS}}

{\large{INSTITUIÇÃO ACADÊMICA}}
\end{center}

\vspace{2cm}

\begin{center}
{\Large\textbf{Título do Trabalho}}
\end{center}

\vspace{1.5cm}

\begin{center}
{\Large{Autor do Trabalho}}
\end{center}

\vspace{4cm}

\begin{flushright}

\begin{minipage}{12cm}

\hrulefill 

Trabalho Final de graduação do curso de Análise e desenvolvimento de Sistemas da Faculdade de Tecnologia do Estado de São Paulo
apresentado como requisito para a obtenção do grau de tecnólogo em Análise e desenvolvimento de sistemas. 

\hrulefill

{\textbf{Orientador: Prof. Dr. Fulano}}
\end{minipage}

\end{flushright}

\setlength{\baselineskip}{0.7\baselineskip}
\vfill

\begin{center}
São Carlos

Abril de 2023
\end{center}

\end{titlepage}

\chapter*{Resumo}

Aqui vai o resumo do seu trabalho...\\
{\textbf{Palavras-chaves:} Matemática, Física e Química}

\chapter*{Agradecimentos}

Aqui vai os agradecimentos para as pessoas mais próximas deste trabalho...

\end{document}